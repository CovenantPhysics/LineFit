\RequirePackage{linefit}
\documentclass[titlepage,12pt]{article}
\begin{document}
oddsidemargin = -1 in

In order to calculate the ohms we took the ratio of the lengths and multiplied it by the known resistors values. This works because we known the voltage drop is the same on both sides of the circuit and we can therefore compare the currents and resistances when the voltage is the same and when we do we get these two equations \(I_{1}R_{x} = I_{2}R_{2}\) and \(I_{1}R_{s} = I_{2}R_{1}\) and when we substitute for the current we get the formula \(R_{x} = \frac{L_{2}}{L_{1}}R_{s}\)

\setbox0=\hbox{
	\shiftleft{3.0}
	\settextsize{12.0}
	\linefitgraphbegin{6.249999999999999}{1.7857142857142856}

	\color[RGB]{153,0,153}
	\putpoint{\symFilledSquare}{2.0}{7.5}
	\putpoint{\symFilledSquare}{3.0}{3.4}
	\putpoint{\symFilledSquare}{4.0}{0.5}
	\putline{1.8}{8.0}{4.142857142857143}{-0.2}
	\putresults{y$_{1}$ = m$_{1}$x + b$_{1}$}{11.468548387096774}{1.4347014925373132}{1.8}{-0.2}
	\putresults{m$_{1}$ = -3.5000$\times10^{1}$}{11.468548387096774}{1.0847014925373133}{1.8}{-0.2}
	\putresults{b$_{1}$ = 14.3000$\times10^{1}$}{11.468548387096774}{0.7347014925373134}{1.8}{-0.2}

	\color{black}
	\drawxaxis{New X-Axis Description}{1.80 2.04 2.28 2.52 2.76 3.00 3.24 3.48 3.72 3.96 4.20}{-0.2}{0.7}
	\drawyaxis{New Y-Axis Description}{-0.20 0.64 1.48 2.32 3.16 4.00 4.84 5.68 6.52 7.35 8.20}{1.8}{1.2083333333333335}
	\putypower{$\times10^{1}$}{1.8}{8.2}{0.35}
	\linefitgraphend
}
\begin{figure}
\begin{center}
\scalebox{1}{\box0}
\caption{New Graph}
\end{center}
\end{figure}

We estimated our errors at one in the last decimal place for our readings on the multimeter because we did not calibrate them. For measuring with the ruler we estimated 0.1 centimeter because of the minor incosistancy in the voltage and the fact that that is how far the meter stick has ticks for.

\end{document}